
\documentclass[preprint,12pt]{elsarticle}

%% Use the option review to obtain double line spacing
%\documentclass[preprint,review,12pt]{elsarticle}

%% Use the options 1p,twocolumn; 3p; 3p,twocolumn; 5p; or 5p,twocolumn
%% for a journal layout:
%% \documentclass[final,1p,times]{elsarticle}
%% \documentclass[final,1p,times,twocolumn]{elsarticle}
%% \documentclass[final,3p,times]{elsarticle}
%% \documentclass[final,3p,times,twocolumn]{elsarticle}
%% \documentclass[final,5p,times]{elsarticle}
%% \documentclass[final,5p,times,twocolumn]{elsarticle}

%% The graphicx package provides the includegraphics command.
\usepackage{graphicx}
%% The amssymb package provides various useful mathematical symbols
\usepackage{amssymb}
%% The amsthm package provides extended theorem environments
%% \usepackage{amsthm}

%% The lineno packages adds line numbers. Start line numbering with
%% \begin{linenumbers}, end it with \end{linenumbers}. Or switch it on
%% for the whole article with \linenumbers after \end{frontmatter}.
\usepackage{lineno}

%% natbib.sty is loaded by default. However, natbib options can be
%% provided with \biboptions{...} command. Following options are
%% valid:

%%   round  -  round parentheses are used (default)
%%   square -  square brackets are used   [option]
%%   curly  -  curly braces are used      {option}
%%   angle  -  angle brackets are used    <option>
%%   semicolon  -  multiple citations separated by semi-colon
%%   colon  - same as semicolon, an earlier confusion
%%   comma  -  separated by comma
%%   numbers-  selects numerical citations
%%   super  -  numerical citations as superscripts
%%   sort   -  sorts multiple citations according to order in ref. list
%%   sort&compress   -  like sort, but also compresses numerical citations
%%   compress - compresses without sorting
%%
%% \biboptions{comma,round}

% \biboptions{}

\journal{NeuroImage}

\begin{document}

\begin{frontmatter}
\title{Unnecessarily Complicated Research Title}

%% use the tnoteref command within \title for footnotes;
%% use the tnotetext command for the associated footnote;
%% use the fnref command within \author or \address for footnotes;
%% use the fntext command for the associated footnote;
%% use the corref command within \author for corresponding author footnotes;
%% use the cortext command for the associated footnote;
%% use the ead command for the email address,
%% and the form \ead[url] for the home page:
%%
%% \title{Title\tnoteref{label1}}
%% \tnotetext[label1]{}
%% \author{Name\corref{cor1}\fnref{label2}}
%% \ead{email address}
%% \ead[url]{home page}
%% \fntext[label2]{}
%% \cortext[cor1]{}
%% \address{Address\fnref{label3}}
%% \fntext[label3]{}



\author[usc]{Anand A. Joshi}
\author[usc]{Jian Li}
\author[usc]{Soyoung Choi}
\author[usc]{Haleh Akrami}
\author[usc]{Jonas Kaplan}
\author[usc]{Richard M. Leahy}

\address[usc]{University of Southern California, Los Angeles, USA}

\begin{abstract}

||AUM||
||Shree Ganeshaya Namaha||

\end{abstract}

\begin{keyword}
fMRI \sep ADHD \sep Resting State \sep group differences
\end{keyword}

\end{frontmatter}
\linenumbers


\section{Introduction}
\label{sec:introduction}

\section{Materials and Methods}
\label{sec:mat_methods}
\subsection{brainsync transform}

\subsection{Statistical analysis}

\subsection{Subsection Two}

\section{Results}
\subsection{ADHD data}

\section{Discussion}

\section{Conclusion}
\label{S:2}


\bibliographystyle{model1-num-names}
\bibliography{sample.bib,zotero_ref.bib}


\end{document}

%%
%% End of file `elsarticle-template-1-num.tex'.